\documentclass{article}

\usepackage[margin=1in]{geometry}

\begin{document}

		\title{A FAST NUMERICAL SOLVER FOR MULTIPHASE FLOW}
		\author{Group : 101}
		\maketitle
	
		\tableofcontents
	\newpage	

\section{Introduction}
\section{Abstract}
\paragraph{We discuss several fully implicit techniques for solving the nonlinear algebraic system arising in an expanded mixed finite element or cell-centered finite difference discretization of two- and three-phase porous media flow. Every outer nonlinear Newton iteration requires solution of a nonsymmetric Jacobian linear system. Two major types of preconditioners, supercoarsening multigrid (SCMG) and two-stage, are developed for the GMRES iteration applied to the solution of the Jacobian system. The SCMG reduces the three-dimensional system to two dimensions using a vertical aggregation followed by a two-dimensional multigrid. The two-stage preconditioners are based on decoupling the system into a pressure and concentration equations. Several pressure preconditioners of different types are described. Extensive numerical results are presented using the integrated parallel reservoir simulator (IPARS) and indicate that these methods have low arithmetical complexity per iteration and good convergence rates.}
\section{Problem Statement}
	\paragraph{Treats surface tension effects as a localized body force. Results are compared with companion simulations carried 	out with the commercial software Fluent.}
	\paragraph{Also, revealing a noticeable improvement in the quality of the solution}
	\paragraph{Above all, caters for reduced computational cost.}

	
	\section{Methodology}
	
\subsection{Mathematical Model}
	\paragraph{In this method we study adjectives flow in a porous medium, that is, we
	neglect gravity and capillary forces and consider viscous flow driven only by
	pressure forces. The fact that viscous flow is unidirectional along streamlines
	will be the key property used to derive our highly efficient upwind solvers.}
	\subsection{Spatial Discretisation}
 	\paragraph{The discretisation in a discontinuous Galerkin method starts with a variational formulation, as in a standard Galerkin method. The difference is that discontinuous Galerkin methods allow discontinuities
	at element interfaces.}
	
\subsection{Treatment of Cycles}
	\paragraph{The number and size of cycles in the discrete fluxes depend on the heterogeneity and on the numerical method used to discretize the pressure equation.}
	
	
\section{Objectives}
          \subsection{Main Objective}
	\paragraph{The goal of this project is to initiate the development of an in-house code at Purdue that can simulate multiphase-flow physics that can exploit state-of-the-art supercomputing architectures}
          \subsection{Other Objective}
	\paragraph{Reordering procedures that will greatly reduce the runtime and the memory requirements needed to compute each time step.}
	
\section{Support needed}

\section{Outcomes}
	\section{Literature Review}
	\section{References}




\end{document}