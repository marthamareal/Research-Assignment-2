\documentclass[a4paper,12pt]{article}
\begin{document}
\title{A fast numerical solver for multiphase flow.}
\author{Group 101}
\maketitle
\section{Introduction.}
In fluid mechanics, multiphase flow is simultaneous flow of materials with different states or phases (i.e. gas, liquid or solid) and materials with different chemical properties but in the same state or phase (i.e. liquid-liquid systems such as oil droplets in water).
\newline
\newline
In nature and environment, rain, snow, fog, avalanches, mud slides, sediment transport, debris flows are all examples of multiphase flow where the behavior of the phases are studied in different fields of natural science.
\newline
\newline
In oil and gas industries, multiphase flow often implies to simultaneous flow of oil, water and gas. The term is also applicable to signify the properties of a flow in some field where there is a chemical injection or various types of inhibitors.
\section{Background to the problem.}
In this paper we discuss algorithms of a fast numerical solver for multiphase flow as well as the proper object oriented implementation of this algorithm. Different time stepping discretization and linearization approaches are discussed. 
\newline 
\newline
Numerical results for one realistic problem are presented and problems involving the multiphase flow, heat transfer, and multicomponent.
Each of the phases is considered to have a separately defined volume fraction (the sum of which is unity) and velocity field. Conservation equations for the flow of each species (perhaps with terms for interchange between the phases) can then be written down straightforwardly.
\section{Problem Statement.}
Mathematical models for multiphase flow usually are formulated at macroscopic level and they are obtained by volume averaging or homogenization methods from microscopic equations.
\newline The resulting models are difficult to solve due to large number of strongly coupled nonlinear differential equations in the systems.
\newline
\newline
In this paper we numerically solve multiphase flow problems and we present and discuss numerical difficulties connected to different approaches for time discretization and for linearization of the governing system of PDEs.
\section{Aim and Objectives.}
\subsection{Main Objective.}
Our goal is to find a fast numerical algorithm for solving problems of multiphase flows thus we have decided to start with not the complicated but flexible and robust approximations.
\newline We try to avoid the time-consuming computation of approximate Riemann solvers and the related characteristic decompositions. Since dimensional splitting and splitting in physical processes also introduce numerical and non-physical errors, we avoid splitting also. 
\subsection{Specific Objectives.}
\begin{itemize}
\item To avoid the time-consuming computation of approximate Riemann solvers and the related characteristic decompositions.
\item To find a fast numerical algorithm for solving problems of multiphase flows. 
\item To describe the numerical method and test its performance. Rather than relying on the periodic Green's function as classical BIE (boundary integral equation) methods do.
\end{itemize}
\end{document}
